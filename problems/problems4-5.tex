\documentclass[a4paper, 10pt]{article}
\usepackage[utf8]{inputenc} % Change according your file encoding
\usepackage{graphicx}
\usepackage{url}
\usepackage{amsmath}
\usepackage{amsthm}
\usepackage[a4paper, left=2cm, right=2cm, top=2cm, bottom=2cm]{geometry}

\newtheorem{obs}{Observation}

%opening
\title{Seminar Report: [seminar ID] (e.g. Paxy)}
\author{\textbf{Name of team members}}
\date{\normalsize\today{}}

\begin{document}

%\maketitle
Carlos Segarra \hfill Thursday, September 26th

\vspace{15pt}

\textbf{\Large Discrete and Algorithmic Geometry: Problems 4 and 5}

\vspace{30pt}

\textbf{\textit{4. Propose an algorithm that, given a point $p$ external to a convex polygon $\mathcal{P}$, finds the point of $\mathcal{P}$ closest to $p$. What happens if instead of finding the closest point we look for the farthest? What if we restrict the search to the vertices of $\mathcal{P}$?}}

\vspace{3pt}

Let $\lbrace q_1, \dots, q_n\rbrace$ be the set of vertices of $\mathcal{P}$ in clock-wise order. And let $f$ be the function s.t. for all $q \in \mathcal{P}$, sends $p$ to $d(q, p)$. Before starting with the proof, let us make two observations:
\begin{obs}
    $f$ is an uni-modal application. Alternatively, its derivative is a linear funcion that has only one zero.
\end{obs}
\begin{obs}
    Let $q_{min}$ and $q_{max}$
\end{obs}

\vspace{10pt}

\begin{center}
    \hrule
\end{center}

\textbf{\textit{5. Propose an algorithm that, given two disjoitn convex polygons, $\mathcal{P}$ and $\mathcal{Q}$, finds the closest pair of points $p \in \mathcal{P}$ and $q \in \mathcal{Q}$.}}

\end{document}
